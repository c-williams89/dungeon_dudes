\documentclass[11pt]{extbook}
% Builds with:
% xelatex writeup.tex

\usepackage{style}
\usepackage{enumitem}
\usepackage{titling}
\usepackage{listings}
\usepackage{color}
\newcommand\tab[1][1cm]{\hspace*{#1}}
\usepackage[utf8]{inputenc}


\title{%
\begin{tabular}{cl}
PROJECT `dungeon\_dudes'\tabularnewline
\large WriteUp\tabularnewline
\large MOD O - OOP 1B\tabularnewline
\small 170D WOBC\tabularnewline
\small Class 23-001\tabularnewline
\end{tabular}
}
\subtitle{\large dungeon\_dudes}

\author{%
\begin{tabular}{cl}
WO1 Josh Kaplan
\end{tabular}
}
\date{\today}

\begin{document}

\maketitle

% \tableofcontents

% \include{a_project_summary}
% \chapter{Project Summary}
\section{\emph{1 Project Summary}}

\tab The project \emph{dungeon\_dudes} is an object-oriented program in Python 
which are to create a Character or Monster for a Dungeon-Crawler style RPG
in accordance with an already existing codebase. The Character or Monster must
match the associated Character Stats and information found in the Manual for 
Dungeon Dudes. This is in order to teach us the appropriate way to integrate 
work into an existing codebase and master the art of git collaboration. 

% \include{b_challenges}
% \chapter{Challenges}
\section{\emph{2 Challenges and Successes}}

\tab As mentioned in class before, I believe the biggest challenge with this 
project (and to falling-in on an existing codebase) is fully knowing and 
understanding what functionality is available from the BaseClasses 
(essentially what is available from the API). 

\tab I had many Successes once I figured out how much of the Fighter Class I 
could reuse and port over for the Cleric. The difficulty lay in the 
implementation of the passive skills, as they aren't necessarily always 
implementable as a class Method like the "active" skills. Some passive skills
were essentially just "bolted-on" to the active skills as necessary.

\tab Unittests were kindof a mixed-bag, as its hard to fully unnittest subsets
of a codebase without also testing the entirety of the codebase that the subset
calls upon.

\tab Git was both a Challenge and a Success, as there were a time where I 
forogt to \verb|git pull| before merging with master, causing massive 
merge-conflicts. However, solving the merge-conflicts and solidifying the right
way to go about group collaboration was a great learning process and a very
valuable skill.

\tab I enjoyed this as a project, from an "applicability" point-of-view, and 
from a "fun" point of view. Subject Matter was different from the usual
projects and felt more in line with the applicability of \verb|nfl|.

\section{\emph{3 Lessons Learned}}

\tab My lessons learned are to take some time to better research what code is 
given in a large OOP codebase (i.e. understanding the API). Lessons learned 
also includes unittesting small subsets of the codebase and mastering git,
despite screwing up the master branch.

\end{document}