\documentclass[11pt]{extbook}
% Builds with:
% xelatex testplan.tex

\usepackage{style}
\usepackage{enumitem}
\usepackage{titling}
\usepackage{listings}
\usepackage{color}
\usepackage{multicol}
\newcommand\tab[1][1cm]{\hspace*{#1}}
\usepackage[utf8]{inputenc}
\usepackage{pdflscape}
\usepackage{changepage}


\title{%
\begin{tabular}{cl}
PROJECT `dungeon\_dudes'\tabularnewline
\large TestPlan\tabularnewline
\large MOD O - OOP 2B\tabularnewline
\small 170D WOBC\tabularnewline
\small Class 23-001\tabularnewline
\end{tabular}
}
\subtitle{\large dungeon\_dudes}

\author{%
\begin{tabular}{cl}
WO1 Josh Kaplan
\end{tabular}
}
\date{\today}

\begin{document}

\maketitle

\section*{\emph{1 Purpose}}

\tab The project \emph{dungeon\_dudes} is an object-oriented program in Python 
which are to create a Character or Monster for a Dungeon-Crawler style RPG
in accordance with an already existing codebase. The Character or Monster must
match the associated Character Stats and information found in the Manual for 
Dungeon Dudes. This is in order to teach us the appropriate way to integrate 
work into an existing codebase and master the art of git collaboration. 

\section*{\emph{2 Test Components}}

\tab This test plan utilizes the \verb|unittest| Python library as 
executed by the tester. Instructions on how to run the tests on the Linux 
Terminal can be found below. 

\emph{Commands can be found in italicized text preceded by a ‘\$>’.}

\emph{Before beginning any tests, ensure you have the repo cloned to the dir of your choice:} \newline
\emph{\$> git clone https://git.cybbh.space/170D/wobc/student-folders/23\_001/kaplan/dungeon\_dudes.git} \newline
\emph{\$> cd dungeon\_dudes}

\subsection*{\emph{2.1 Automated Tests}}

The Automated tests cover the each of the key functions 
within \emph{dungeon\_dudes}. This includes:

\begin{multicols}{2}
    \begin{itemize}
        \item Cleric.\_\_init\_\_().Instance
        \item Cleric.\_\_init\_\_().Skills\_dict
        \item Cleric.\_\_init\_\_().Passive\_skills
        \item Cleric.\_\_init\_\_().\_avenged
        \item Cleric.\_\_init\_\_().\_nodamage
        \item Cleric.\_\_init\_\_().\_halfdamage
        \item Cleric.\_\_init\_\_().\_retribution
        \item Cleric.\_\_init\_\_().hit\_points
        \item Cleric.\_\_init\_\_().special
        % 
        \item Cleric.level\_up().level
        \item Cleric.level\_up().hit\_points
        \item Cleric.level\_up().special

    \end{itemize}
    \end{multicols}

Each function is run with good of data to ensure 
complete functionality of each function. Once you are ready to run the tests, 
run the following command from the within the repo directory: 
\newline
\emph{\$> python3 -m unittest kaplan\_test/test\_cleric.py} \newline 

The expected output can be found below in Section 3.
\newpage
\subsection*{\emph{2.2 Manual Tests}}

There are no manual tests for \emph{dungeon\_dudes}.

\begin{landscape}
\section*{\emph{3 Expected Output}}
\begin{verbatim}
    .test the Cleric has reached level 2!
    test the Cleric has reached level 3!

    New Skill - Radiance: Deal Holy damage to all enemies for Intelligence + AtkPower
    test the Cleric has reached level 4!
    test the Cleric has reached level 5!
    New Skill - Divine Blessing: Passive: When you take action in combat, you will recover 10percent 
                of your maximium HP when at full Mana, and 10percent of your maximum mana when at 
                full Hit Points.
    .
    ----------------------------------------------------------------------
    Ran 2 tests in 0.000s

    OK
\end{verbatim}
\end{landscape}
\end{document}
